
\documentclass[10pt, a4paper]{beamer}

\usetheme{Berkeley}
\usecolortheme{sidebartab}

\begin{document}
	\setbeamertemplate{sidebar left}{}
	\title{Progress Presentation-I}
	\subtitle{e-Yantra Summer Intership-2016 \\ Modular Robots}
	\author{Srijal Poojari\\Madhav Wagh\\
	\textbf{Mentor}: Pushkar Raj}
	\institute{IIT Bombay}
	\date{\today}
	%\addtobeamertemplate{sidebar left}{}{\includegraphics[scale = 0.3]{logowithtext.png}}
	\frame{\titlepage}

\setbeamertemplate{sidebar left}[sidebar theme]
\section{Overview of Project}
\begin{frame}{Overview of Project}
	Give following details: \\
	\begin{itemize}
		\item \textbf{Project Name}:  Modular Robots
		\item \textbf{Objective} 
		\begin{enumerate}
		\item To build a Self-reconfigurable autonomous 	     		robot which can deliberately change shape by 				reorganizing connectivity between the modules.
		\linebreak
		\item To add sensors to the robot and make it smart. 		\\(To sense and take action according to the 				  environment)
		\linebreak
		\end{enumerate}
		\item \textbf{Deliverables}
		\begin{enumerate}
		 \item A stable modular robot which is able to 					  change its shape upon the need of the 					  environment\linebreak
		\item Code and Documentation of each Task (1-6)
		\end{enumerate}
	\end{itemize}
\end{frame}

\section{Overview of Task}
\begin{frame}{Overview of Task}
	\begin{itemize}
		\item \textbf{List Of Key Tasks with Deadlines} 
		
	\end{itemize}
	
	\begin{center}
    \begin{tabular}{ | l | p{5cm} | l |}
    \hline
     \textbf{Task No.} & \textbf{Task} & \textbf{Deadline}\\ \hline
     1 & Getting Familiar with existing models of Modular Robots & 2 days  \\ \hline
     
     2 & Interfacing Arduino IDE with Servo, Bluetooth and Sensor & 3 days  \\ \hline
     
     3 & Testing and selecting appropriate sensors to be added in the module & 2 days  \\ \hline
     
     4 & Make design changes in the modules for accommodating sensors.  & 4 days  \\ \hline
     
    
    
    
    \end{tabular}
\end{center}

   

	
\end{frame}


\section{Overview of Task}
\begin{frame}{Overview of Task}
	
	\begin{center}
    \begin{tabular}{ | l | p{5cm} | l |}
    \hline
     \textbf{Task No.} & \textbf{Task} & \textbf{Deadline}\\ \hline
     
     
     5 & Assembling all the selected parts. Four robotic modules need to be produced & 4 days  \\ \hline
     
     6 & Applying algorithm to check different types of motion (Wheel, Snake, Ladder)  & 7 days  \\ \hline
     
     7 & Autonomous Obstacle Avoidance using sensor detection and self-reconfiguration & 6 days  \\ \hline
     
     8 & Code \& Documentation & 6 days  \\ \hline
    
    
    \end{tabular}
\end{center}

   

	
\end{frame}

\section{Task Accomplised}
\begin{frame}{Task Accomplised}
	\begin{itemize}
		\item \textbf{Task-1:} Got Familiar with existing models, selected the most suitable model based on efficiency, expandability and time constraints.\linebreak
		\item \textbf{Task-2:} Interfaced Arduino Nano with Servo Motors, Bluetooth and Sensor\linebreak
		\item \textbf{Task-3:} Testing and selecting appropriate sensors to add in the module.
	Two Sensors were successfully interfaced and calibrated:
	\\1) Sharp Sensor         
	\\2) Laser TOF Sensor (selected based on size, range and accuracy)\linebreak



		
	\end{itemize}
\end{frame}

\section{Task Accomplised}
\begin{frame}{Task Accomplised}
	\begin{itemize}
		

		\item  \textbf{Task-4:}  Studied the design and made design changes to module to change the hole size as to fit the available screw dimension. \linebreak

		\item \textbf{Task-5:}  Simulated the movements of the designed modular robot.
		\begin{itemize}
		
		
\item Interfaced Laser TOF sensor in simulation environment and took feedback for reorganization.                                                                                                                              \item Also scripted it in LUA to overcome obstacles. (Attached video)
        \linebreak After successful simulation of the design the parts are given for printing.
        \end{itemize}

		
	\end{itemize}
\end{frame}



\section{Challenges Faced}
\begin{frame}{Challenges Faced}
	\begin{itemize}
		\item Appropriate screw (2mm x 4mm Flathead) not available, so had to change the 3D CAD design
		\item Selection of Sensors which would fit the free space available in the design, and also serve the purpose of successful obstacle detection.

		\item Coding on V-REP using LUA. The V-REP script flow is time dependent. (Add flow diagram)

 
	\end{itemize}
\end{frame}

\section{Future Plans}
\begin{frame}{Future Plans}
	\begin{itemize}
		\item All printed parts assembled. Four Robotic modules to be assembled
		\item Applying algorithm to check different type of motion (Wheel, Snake)

		\item  Begin with Autonomous obstacle avoidance using sensor detection.

	\end{itemize}
\end{frame}


\section{Thank You}
\begin{frame}{Thank You}
	\centering THANK YOU !!!
\end{frame}
\end{document}
